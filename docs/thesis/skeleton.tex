% UG project example file, February 2022
%   A minior change in citation, September 2023 [HS]
% Do not change the first two lines of code, except you may delete "logo," if causing problems.
% Understand any problems and seek approval before assuming it's ok to remove ugcheck.
\documentclass[logo,bsc,singlespacing,parskip]{infthesis}
\usepackage{ugcheck}

% Include any packages you need below, but don't include any that change the page
% layout or style of the dissertation. By including the ugcheck package above,
% you should catch most accidental changes of page layout though.

\usepackage{microtype} % recommended, but you can remove if it causes problems
\usepackage{natbib} % recommended for citations

\begin{document}
\begin{preliminary}

\title{End-of-Turn Detection}

\author{Sean Leishman}

% CHOOSE YOUR DEGREE a):
% please leave just one of the following un-commented
%\course{Artificial Intelligence}
%\course{Artificial Intelligence and Computer Science}
%\course{Artificial Intelligence and Mathematics}
%\course{Artificial Intelligence and Software Engineering}
%\course{Cognitive Science}
%\course{Computer Science}
%\course{Computer Science and Management Science}
%\course{Computer Science and Mathematics}
%\course{Computer Science and Physics}
%\course{Software Engineering}
\course{Master of Informatics} % MInf students

% CHOOSE YOUR DEGREE b):
% please leave just one of the following un-commented
\project{MInf Project (Part 1) Report}  % 4th year MInf students
%\project{MInf Project (Part 2) Report}  % 5th year MInf students
%\project{4th Year Project Report}        % all other UG4 students


\date{\today}

\abstract{
This skeleton demonstrates how to use the \texttt{infthesis} style for
undergraduate dissertations in the School of Informatics. It also emphasises the
page limit, and that you must not deviate from the required style.
The file \texttt{skeleton.tex} generates this document and should be used as a
starting point for your thesis. Replace this abstract text with a concise
summary of your report.
}

\maketitle

\newenvironment{ethics}
   {\begin{frontenv}{Research Ethics Approval}{\LARGE}}
   {\end{frontenv}\newpage}

\begin{ethics}
\textbf{Instructions:} \emph{Agree with your supervisor which
statement you need to include. Then delete the statement that you are not using,
and the instructions in italics.\\
\textbf{Either complete and include this statement:}}\\ % DELETE THESE INSTRUCTIONS
%
% IF ETHICS APPROVAL WAS REQUIRED:
This project obtained approval from the Informatics Research Ethics committee.\\
Ethics application number: ???\\
Date when approval was obtained: YYYY-MM-DD\\
%
\emph{[If the project required human participants, edit as appropriate, otherwise delete:]}\\ % DELETE THIS LINE
The participants' information sheet and a consent form are included in the appendix.\\
%
% IF ETHICS APPROVAL WAS NOT REQUIRED:
\textbf{\emph{Or include this statement:}}\\ % DELETE THIS LINE
This project was planned in accordance with the Informatics Research
Ethics policy. It did not involve any aspects that required approval
from the Informatics Research Ethics committee.

\standarddeclaration
\end{ethics}


\begin{acknowledgements}
Any acknowledgements go here.
\end{acknowledgements}


\tableofcontents
\end{preliminary}


\chapter{Introduction}

The preliminary material of your report should contain:
\begin{itemize}
\item
The title page.
\item
An abstract page.
\item
Declaration of ethics and own work.
\item
Optionally an acknowledgements page.
\item
The table of contents.
\end{itemize}

As in this example \texttt{skeleton.tex}, the above material should be
included between:
\begin{verbatim}
\begin{preliminary}
    ...
\end{preliminary}
\end{verbatim}
This style file uses roman numeral page numbers for the preliminary material.

The main content of the dissertation, starting with the first chapter,
starts with page~1. \emph{\textbf{The main content must not go beyond page~40.}}

The report then contains a bibliography and any appendices, which may go beyond
page~40. The appendices are only for any supporting material that's important to
go on record. However, you cannot assume markers of dissertations will read them.

You may not change the dissertation format (e.g., reduce the font size, change
the margins, or reduce the line spacing from the default single spacing). Be
careful if you copy-paste packages into your document preamble from elsewhere.
Some \LaTeX{} packages, such as \texttt{fullpage} or \texttt{savetrees}, change
the margins of your document. Do not include them!

Over-length or incorrectly-formatted dissertations will not be accepted and you
would have to modify your dissertation and resubmit. You cannot assume we will
check your submission before the final deadline and if it requires resubmission
after the deadline to conform to the page and style requirements you will be
subject to the usual late penalties based on your final submission time.

\section{Turn-taking Prediction}
\subsection{From the Conversation Analysis Perspective}
The most widely accepted model of turn-taking originates from \cite{Sacks1974}. They created their model based off a clear pattern in behaviour during social interaction. With this in mind they determine that the behaviour exhibits a joint and coordinated effortto determine turns while attempting to have a minimal number of turns. Although there is this effort, turn-taking organisation is not preplanned and it is coordinated in a manner that changes along the course of a conversation. During a dialogue \cite{Sacks1974} note that convservation is dominated by a single speaker at a time although multiple speakers at one time do occur commonly but they are brief. When the speaker switches, the transitions are characterised with no gap or overlap, most commonly, but also with a small overlap or gap. From these observations \cite{Sacks1974} suggested a series of definitions for these units of speech and rules to describe their interaction. Turn-taking is separated into units of speech called \textit{Turn-Constructional-Units (TCU)} where predominately one speaker is speaking and after each TCU there is a \textit{Transition-Relevant-Phase (TRP)} where the current speaker switches (turn-shift). 
\cite{Sacks1974} noted that a turn-shift can but does not have to occur. The same could be said in reverse, that a turn-shift can occur when there is no TRP. At each TRP  \cite{Sacks1974} noted the turn-allocation techniques used to determine the next speaker. These techniques can be grouped into two categories: 'current selects next' and 'self-selection'. The current speaker selection could employ techniques such as gaze but for self selection it is simply the case that the first to start has the turn if the current speaker exhibits a non-selection. The current speaker is also able to select themselves if no other particpant self-selects. 

The development of speech technology has allowed for the automatic analysis of turn-taking by employing statistical analyis of a large corpora. \cite{LevTor2015} provides a systemic overview of the properties first noted by \cite{Sacks1974} and produced their own model of turn-taking where: turns are mostly short (mean=1680ms, median=1227ms); gaps in a speaker's current utterance are longer than those between speakers suggesting that the next speaker has the right to speak prior to the current speaker and also that overlaps are more common at turn transitions and within turns. They also develop a psycholinguistic model for turn-taking that backs the claim made by \cite{Sacks1974} that turn-taking requires projection. 
<F8>
\cite{LevTor2015} turn transitions typically occur between -100ms and 500ms. This is proven probelmatic, if we employ a reaction-based system, where studies have shown that language production latenices range from 600ms - 1500ms \cite{IndLev2004, Bates2003} and are at minimum 200ms (for a prepared vowel \cite{Fry1975}). A combination with a silence becoming reognisable (180ms \cite{}), the reaction to a silence (100ms) and language productions would result in at least a 550ms turn transition, outside the typical range. 

As such, the question remains, how, if a non-speaker wishes to self-select as described in \cite{Sacks1974}, can they predict a TRP if they are not simply responding to silence as claimed by \cite{HelEdl2010}. 

\subsubsection{Turn-taking Cues} 
The first systematic studies of turning taking cues came from \cite{Duncan1972, Duncan1974} where they identified a number of turn-taking cues that signal turn-completion across a few different domains. Syntacially complete phrases, phrase-final intonation, termination of hand gesticulation were all identified as turn-taking cues. In literature, it appears that syntactic and prosodic features tend to have received the most focus as turn-taking cues. \cite{Sacks1974} first argued that syntax and semantics played a more influential role in prediction than a prosodic feature such as intonation. This idea was investigated further by \cite{Ford1996} who concluded that intonational completion
re than previously considered, as compared to linguistic features such as syntactic completeness. \cite{Ford1996} defined an utterance as being \texttt{syntactically complete} if "in it's discourse context, it could be intepreted as a complete clause, that is, without an overt or directly reoverable predicte without considering intonation of interactional import". They also define another feature of \textt{pragmatic completness} that is judged as the completion of intonational and conversational action sequencing. (Explain conversational action sequencing?). \cite{Ford1996} found that points of both pragmatic and intonational complete are nearly always syntactic complete points however the reverse is not always true. As such, they defined these points of pragmatic, intonational and syntactic completeness as a \textt{Complex Transition Relevance Place}, where a 71\% of speaker changes occur at CRTPs. (CRITIQUE). \cite{Ford1996} 


\chapter{Your next chapter}

A dissertation usually contains several chapters.

\chapter{Conclusions}

\section{Final Reminder}

The body of your dissertation, before the references and any appendices,
\emph{must} finish by page~40. The introduction, after preliminary material,
should have started on page~1.

You may not change the dissertation format (e.g., reduce the font size, change
the margins, or reduce the line spacing from the default single spacing). Be
careful if you copy-paste packages into your document preamble from elsewhere.
Some \LaTeX{} packages, such as \texttt{fullpage} or \texttt{savetrees}, change
the margins of your document. Do not include them!

Over-length or incorrectly-formatted dissertations will not be accepted and you
would have to modify your dissertation and resubmit. You cannot assume we will
check your submission before the final deadline and if it requires resubmission
after the deadline to conform to the page and style requirements you will be
subject to the usual late penalties based on your final submission time.

% \bibliographystyle{plain}
\bibliographystyle{plainnat}
\bibliography{mybibfile}


% You may delete everything from \appendix up to \end{document} if you don't need it.
\appendix

\chapter{First appendix}

\section{First section}

Any appendices, including any required ethics information, should be included
after the references.

Markers do not have to consider appendices. Make sure that your contributions
are made clear in the main body of the dissertation (within the page limit).

\chapter{Participants' information sheet}

If you had human participants, include key information that they were given in
an appendix, and point to it from the ethics declaration.

\chapter{Participants' consent form}

If you had human participants, include information about how consent was
gathered in an appendix, and point to it from the ethics declaration.
This information is often a copy of a consent form.


\end{document}
